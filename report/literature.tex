\begin{thebibliography}{99}
\bibitem{Kormen}
Маннинг, Рагхаван, Шютце. \textit{Введение в информационный поиск} --- Издательский дом \enquote{Вильямс}, 2011. Перевод с английского: доктор физ.-мат. наук Д.\,А.\, Клюшина --- 528 с. (ISBN 978-5-8459-1623-4 (рус.)).


\bibitem{IR_gist}
\href{https://gist.github.com/AaradhyaSaxena/e20e7dd1556683523f58fe3d5cc463d2}{AaradhyaSaxena, \textit{Introduction to Information Retrieval} --- GitHub Gist, 2025.} (дата обращения: 10.12.2025).

\bibitem{Zipfa_Pitcher}
\href{https://pitcher.agency/blog/zakon-cipfa}{Как поисковые системы оценивают тексты: закон Ципфа, индекс туманности и прочее --- Pitcher.agency, 2025.} (дата обращения: 23.12.2025).

\bibitem{Habr_HH_Search}
\href{https://habr.com/ru/companies/hh/articles/413261/}{Бичук А. \textit{Как устроен поиск} --- Хабр, 05.06.2018.} (дата обращения: 10.12.2025).

\bibitem{Habr_Boolean}
\href{https://habr.com/ru/articles/716968}{Веприкова Д. \textit{Boolean search для чайников и кофейников} --- Хабр, 14.02.2023.} (дата обращения: 11.12.2025).


\end{thebibliography}
\pagebreak