\begin{thebibliography}{99}
\bibitem{Kormen}
Маннинг, Рагхаван, Шютце. \textit{Введение в информационный поиск} --- Издательский дом \enquote{Вильямс}, 2011. Перевод с английского: доктор физ.-мат. наук Д.\,А.\, Клюшина --- 528 с. (ISBN 978-5-8459-1623-4 (рус.)).


\bibitem{intro_IR}
Introduction to Information Retrieval [Электронный ресурс] // GitHub Gist. — URL: {https://gist.github.com/AaradhyaSaxena/e20e7dd1556683523f58fe3d5cc463d2} (дата обращения: 28.12.2025).

\bibitem{law_zipf}
Как поисковые системы оценивают тексты: закон Ципфа, индекс туманности и прочее [Электронный ресурс] // Pitcher.agency. — URL: {https://pitcher.agency/blog/zakon-cipfa} (дата обращения: 28.12.2025).

\bibitem{habr_sportmaster}
Основы полнотекстового поиска в ElasticSearch. Часть вторая [Электронный ресурс] // Хабр: Sportmaster Lab. — URL: {https://habr.com/ru/companies/sportmaster_lab/articles/756270/} (дата обращения: 28.12.2025).

\bibitem{habr_hh_search}
Как устроен поиск [Электронный ресурс] // Хабр: HeadHunter. — URL: {https://habr.com/ru/companies/hh/articles/413261/} (дата обращения: 28.12.2025).


\end{thebibliography}



\pagebreak